%%%%%%%%%%%%%%%%%%%%%%%%%%%%%%%%%%%%%%%%%
% Programming/Coding Assignment
% LaTeX Template
%
% This template has been downloaded from:
% http://www.latextemplates.com
%
% Original author:
% Ted Pavlic (http://www.tedpavlic.com)
%
% Note:
% The \lipsum[#] commands throughout this template generate dummy text
% to fill the template out. These commands should all be removed when 
% writing assignment content.
%
% This template uses a Perl script as an example snippet of code, most other
% languages are also usable. Configure them in the "CODE INCLUSION 
% CONFIGURATION" section.
%
%%%%%%%%%%%%%%%%%%%%%%%%%%%%%%%%%%%%%%%%%

%----------------------------------------------------------------------------------------
%	PACKAGES AND OTHER DOCUMENT CONFIGURATIONS
%----------------------------------------------------------------------------------------

\documentclass{article}

\usepackage{fancyhdr} % Required for custom headers
\usepackage{lastpage} % Required to determine the last page for the footer
\usepackage{extramarks} % Required for headers and footers
\usepackage[usenames,dvipsnames]{color} % Required for custom colors
%\usepackage{graphicx}% % Required to insert images
\usepackage{listings} % Required for insertion of code
\usepackage{courier} % Required for the courier font
 \usepackage[pdftex]{graphicx}
  \graphicspath{{figures/}}
   \DeclareGraphicsExtensions{.pdf,.jpg,.png}
  \usepackage[caption=false,font=footnotesize]{subfig}
%\usepackage[cmex10]{amsmath}
\usepackage{amsmath,amssymb}

\interdisplaylinepenalty=2500

% Margins
\topmargin=-0.45in
\evensidemargin=0in
\oddsidemargin=0in
\textwidth=6.5in
\textheight=9.0in
\headsep=0.25in

\linespread{1.1} % Line spacing

% Set up the header and footer
\pagestyle{fancy}
\lhead{\hmwkAuthorName} % Top left header
%\chead{\hmwkClass\ (\hmwkClassInstructor\ \hmwkClassTime): \hmwkTitle} % Top center head
\rhead{\firstxmark} % Top right header
\lfoot{\lastxmark} % Bottom left footer
\cfoot{} % Bottom center footer
\rfoot{Page\ \thepage\ of\ \protect\pageref{LastPage}} % Bottom right footer
\renewcommand\headrulewidth{0.4pt} % Size of the header rule
\renewcommand\footrulewidth{0.4pt} % Size of the footer rule

\setlength\parindent{0pt} % Removes all indentation from paragraphs

%----------------------------------------------------------------------------------------
%	CODE INCLUSION CONFIGURATION
%----------------------------------------------------------------------------------------
%
\usepackage{listings}
\usepackage{color}

\definecolor{dkgreen}{rgb}{0,0.6,0}
\definecolor{gray}{rgb}{0.5,0.5,0.5}
\definecolor{mauve}{rgb}{0.58,0,0.82}

\lstset{frame=tb,
  language=Python,
  aboveskip=3mm,
  belowskip=3mm,
  showstringspaces=false,
  columns=flexible,
  basicstyle={\small\ttfamily},
  numbers=none,
  numberstyle=\tiny\color{gray},
  keywordstyle=\color{blue},
  commentstyle=\color{dkgreen},
  stringstyle=\color{mauve},
  breaklines=true,
  breakatwhitespace=true,
  tabsize=3
}


%----------------------------------------------------------------------------------------
%	DOCUMENT STRUCTURE COMMANDS
%	Skip this unless you know what you're doing
%----------------------------------------------------------------------------------------

%% Header and footer for when a page split occurs within a problem environment
\newcommand{\enterProblemHeader}[1]{
\nobreak\extramarks{#1}{#1 continued on next page\ldots}\nobreak
\nobreak\extramarks{#1 (continued)}{#1 continued on next page\ldots}\nobreak
}

% Header and footer for when a page split occurs between problem environments
\newcommand{\exitProblemHeader}[1]{
\nobreak\extramarks{#1 (continued)}{#1 continued on next page\ldots}\nobreak
\nobreak\extramarks{#1}{}\nobreak
}

\setcounter{secnumdepth}{0} % Removes default section numbers
\newcounter{homeworkProblemCounter} % Creates a counter to keep track of the number of problems

\newcommand{\homeworkProblemName}{}
\newenvironment{homeworkProblem}[1][Task \arabic{homeworkProblemCounter}]{ % Makes a new environment called homeworkProblem which takes 1 argument (custom name) but the default is "Problem #"
\stepcounter{homeworkProblemCounter} % Increase counter for number of problems
\renewcommand{\homeworkProblemName}{#1} % Assign \homeworkProblemName the name of the problem
\section{\homeworkProblemName} % Make a section in the document with the custom problem count
\enterProblemHeader{\homeworkProblemName} % Header and footer within the environment
}{
\exitProblemHeader{\homeworkProblemName} % Header and footer after the environment
}

\newcommand{\problemAnswer}[1]{ % Defines the problem answer command with the content as the only argument
\noindent\framebox[\columnwidth][c]{\begin{minipage}{0.98\columnwidth}#1\end{minipage}} % Makes the box around the problem answer and puts the content inside
}

\newcommand{\homeworkSectionName}{}
\newenvironment{homeworkSection}[1]{ % New environment for sections within homework problems, takes 1 argument - the name of the section
\renewcommand{\homeworkSectionName}{#1} % Assign \homeworkSectionName to the name of the section from the environment argument
\subsection{\homeworkSectionName} % Make a subsection with the custom name of the subsection
\enterProblemHeader{\homeworkProblemName\ [\homeworkSectionName]} % Header and footer within the environment
}{
\enterProblemHeader{\homeworkProblemName} % Header and footer after the environment
}

%----------------------------------------------------------------------------------------
%	NAME AND CLASS SECTION
%----------------------------------------------------------------------------------------

\newcommand{\hmwkTitle}{Coursework \#1} % Assignment title
\newcommand{\hmwkDueDate}{Friday,\ February 12,\ 2016} % Due date
\newcommand{\hmwkClass}{COMPM080} % Course/class
\newcommand{\hmwkAuthorName}{Maria Ruxandra Robu} % Your name

%----------------------------------------------------------------------------------------
%	TITLE PAGE
%----------------------------------------------------------------------------------------

\title{
\vspace{2in}
\textmd{\textbf{\hmwkClass:\ \hmwkTitle}}\\
\normalsize\vspace{0.1in}\small{Due\ on\ \hmwkDueDate}\\
%\vspace{0.1in}\large{\textit{\hmwkClassInstructor\ \hmwkClassTime}
\vspace{3in}
}

\author{\textbf{\hmwkAuthorName  - 14042500}}
\date{} % Insert date here if you want it to appear below your name

%----------------------------------------------------------------------------------------

\begin{document}

\maketitle

%----------------------------------------------------------------------------------------
%	TABLE OF CONTENTS
%----------------------------------------------------------------------------------------

%\setcounter{tocdepth}{1} % Uncomment this line if you don't want subsections listed in the ToC

%\newpage
%\tableofcontents
\newpage

Introduction
%----------------------------------------------------------------------------------------
%	PROBLEM 1
%----------------------------------------------------------------------------------------

% To have just one problem per page, simply put a \clearpage after each problem

\begin{homeworkProblem}

\textbf{Basic ICP algorithm}\\


%%%%%%%%%%%%%%%%%%%%%%%%%%%%%%%%%%%%%%%%
\begin{figure*}[h]
\centering
\subfloat[]{\includegraphics[height=2in]{task2_noAlign_before}%
\label{fig_t2_noalign_b}}
\hfil
\subfloat[]{\includegraphics[height=2in]{task2_noAlign_after}%
\label{fig_t2_noalign_a}}
\caption{Task 2 - no initial alignment - 27 iterations - error = 0.11854}
\label{fig:t2_noalign}
\end{figure*}
%%%%%%%%%%%%%%%%%%%%%%%%%%%%%%%%%%%%%%%%

%%%%%%%%%%%%%%%%%%%%%%%%%%%%%%%%%%%%%%%%
\begin{figure*}[h]
\centering
\subfloat[]{\includegraphics[height=2in]{task2_align_before}%
\label{fig_t2_align_b}}
\hfil
\subfloat[]{\includegraphics[height=2in]{task2_align_after}%
\label{fig_t2_align_a}}
\caption{Task 2 - no initial alignment - 11 iterations - error = 0.116692}
\label{fig:t2_align}
\end{figure*}
%%%%%%%%%%%%%%%%%%%%%%%%%%%%%%%%%%%%%%%%



\end{homeworkProblem}

%\clearpage
%----------------------------------------------------------------------------------------
%	PROBLEM 2
%----------------------------------------------------------------------------------------

\begin{homeworkProblem}

\textbf{Vary rotation}\\

The mesh M1 was centered at its origin by subtracting the mean $\bar{p}$ from all its points $p$, as shown in Figure \ref{fig_t3_m1_centered}. Once its position is set, incremental rotations are applied to the mesh in the x, y and z axes. Figure  \ref{fig_t3_m3_rot} shows an example for 3 rotations along the x axis. Once we have the initial M1 and the rotated M3, ICP is ran for each iteration. Figure \ref{fig_t3_m3_goodAlign} illustrates the results for the 3 rotations in the example. \\

%\begin{equation}
%\tilde{p} = p - \bar{p}
%\end{equation}



%%%%%%%%%%%%%%%%%%%%%%%%%%%%%%%%%%%%%%%%
\begin{figure*}[h]
\centering
\subfloat[]{\includegraphics[height=1.5in]{t3_m1_centered}%
\label{fig_t3_m1_centered}}
\hfil
\subfloat[]{\includegraphics[height=1.5in]{t3_rotx}%
\label{fig_t3_m3_rot}}
\hfil
\subfloat[]{\includegraphics[height=1.5in]{t3_goodresults}%
\label{fig_t3_m3_goodAlign}}
\caption{Task 3 - Introduction}
\label{fig:t3_intro}
\end{figure*}
%%%%%%%%%%%%%%%%%%%%%%%%%%%%%%%%%%%%%%%%


The parameters used for this section are:
\begin{itemize}
  \item number of different rotations =  25
  \item interval of degrees [-50, 50]
  \item maximum number of iterations for ICP = 50
  \item error threshold for ICP = 1e-04
  \item threshold for the bad points rejection = 70\%
\end{itemize}


%%%%%%%%%%%%%%%%%%%%%%%%%%%%%%%%%%%%%%%%
\begin{figure*}[h]
\centering
\subfloat[]{\includegraphics[height=1.5in]{task3_xaxis}%
\label{fig_t3_conv_x}}
\hfil
\subfloat[]{\includegraphics[height=1.5in]{task3_yaxis}%
\label{fig_t3_conv_y}}
\hfil
\subfloat[]{\includegraphics[height=1.5in]{task3_zaxis}%
\label{fig_t3_conv_z}}
\caption{Task 3 - Convergence Analysis for the 3 axes of rotation}
\label{fig:t3_conv}
\end{figure*}
%%%%%%%%%%%%%%%%%%%%%%%%%%%%%%%%%%%%%%%%

For each iteration, the final error and number of iterations were saved in a file and plotted with Matlab. The figures \ref{fig:t3_conv} show the basins of convergence for the ICP algorithm by varying the rotation angle on the 3 axes. The minimum error obtained (1.48e-11) was for the z axis at -4 degrees and the maximum (33.64) on the y axis for -33 degrees. 

%\begin{lstlisting}
%from mock import patch
%@patch('greengraph.map.Map')
%def getMapMock(mockMap):
%	# patch decorator
%	mockMap.pixels=someTestImage
%	numGreenPix=mockMap.count_green()
%\end{lstlisting}


%\problemAnswer{
%\begin{center}
%%\includegraphics[width=0.75\columnwidth]{example_figure} % Example image
%\end{center}
%
%\lipsum[3-5]
%}
\end{homeworkProblem}

%----------------------------------------------------------------------------------------

\begin{homeworkProblem}


\textbf{Noise}\\

mnoisy2 - 0.01
bunny on fire - orange 0.00055, red noise 0.001


between 0.0001 and 0.005 - 15 samples

%%%%%%%%%%%%%%%%%%%%%%%%%%%%%%%%%%%%%%%%
\begin{figure*}[h]
\centering
\subfloat[]{\includegraphics[height=2in]{t4_noiseBoth}%
\label{fig_t4_noiseBoth}}
\caption{Task 4 - Bunny with different levels of noise. Orange corresponds to 55e-04 standard deviation and red to 1e-03 standard deviation}
\label{fig:t4_noisyEx}
\end{figure*}
%%%%%%%%%%%%%%%%%%%%%%%%%%%%%%%%%%%%%%%%

%%%%%%%%%%%%%%%%%%%%%%%%%%%%%%%%%%%%%%%%
\begin{figure*}[h]
\centering
\subfloat[]{\includegraphics[height=2.3in]{task4_noiseComp}%
\label{fig_t4_noisyComp}}
\hfil
\subfloat[]{\includegraphics[height=2.3in]{task4_noiseNonoise}%
\label{fig_t4_noiseNoNoise}}
\caption{Task 4 - Noise analysis}
\label{fig:t4_noisyAnalysis}
\end{figure*}
%%%%%%%%%%%%%%%%%%%%%%%%%%%%%%%%%%%%%%%%

\end{homeworkProblem}


%----------------------------------------------------------------------------------------

\begin{homeworkProblem}
  
\textbf{Subsampling}\\


%%%%%%%%%%%%%%%%%%%%%%%%%%%%%%%%%%%%%%%%
\begin{figure*}[h]
\centering
\subfloat[]{\includegraphics[height=2.3in]{task5_subsampled35}%
\label{fig_t5_subsampled35}}
\caption{Task 5 - Example of subsampling - 35\%}
\label{fig:t5_subsEx}
\end{figure*}
%%%%%%%%%%%%%%%%%%%%%%%%%%%%%%%%%%%%%%%%


%%%%%%%%%%%%%%%%%%%%%%%%%%%%%%%%%%%%%%%%
\begin{figure*}[h]
\centering
\subfloat[]{\includegraphics[height=2.3in]{task5_errComp}%
\label{fig_t5_subsErrComp}}
\hfil
\subfloat[]{\includegraphics[height=2.3in]{task5_iterationsComp}%
\label{fig_t5_subsIterComp}}
\caption{Task 4 - Noise analysis}
\label{fig:t5_subsampling}
\end{figure*}
%%%%%%%%%%%%%%%%%%%%%%%%%%%%%%%%%%%%%%%%

\end{homeworkProblem}


%----------------------------------------------------------------------------------------

\begin{homeworkProblem}
  
\textbf{Multi-body}\\


\end{homeworkProblem}


%----------------------------------------------------------------------------------------
\begin{homeworkProblem}
  
\textbf{Normal estimation + improved icp}\\


\end{homeworkProblem}


%----------------------------------------------------------------------------------------



\end{document}